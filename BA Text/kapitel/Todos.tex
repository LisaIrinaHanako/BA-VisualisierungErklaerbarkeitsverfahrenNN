% Todos
\section{Todos}

\begin{itemize}
	\item Einleitung: Motivation, Ziele  \done{Einleitung: Ziele umschreiben}{}
	\item Was passiert in dieser Arbeit?
	\done{Einleitung: Vorgehen erläutern}{}
	\done{Kapitel 2 benennen}{}
	\item Erläuterung der Ansätze (Oberthemen; Einführung in das Kapitel)
	\done{Kapitel 2: Grundlagen}{}
	\done{Kapitel 2: NN Aufbau}{}
	\Todo{Kapitel 2: Backpropagation}{}
	\done{Kapitel 2: Ansatzerklärungen übernehmen}{}
	\done{Kapitel 2: Verbindung zu Blackbox öffnenden Ansätzen herstellen}{}
	\begin{itemize}
		\item Surrogat
		\done{Kapitel 2.1: Surrogat detaillierter}{}
		\item Counterfactual
		\done{Kapitel 2.2: Counterfactual detaillierter}{}
		\item Feature Contribution
		\done{Kapitel 2.3: Feature Contribution detaillierter}{}
	\end{itemize} 
	\item Erklärung im speziellen: Surrogat
	\done{Kapitel 3.1: Surrogatmodell anlegen}{}
	\begin{itemize}
		\item DT Theorie
		\done{Kapitel 3.1.1: DT übernehmen}{}
		\done{Kapitel 3.1.1: DT verwendete Formeln erklären}{}
		\Todo{Kapitel 3.1.1: Grafiken ergänzen}{}
		\item DT Implementierung --> selfmade
		\Todo{Kapitel 3.1.2: Implementierung beenden}{}
		\Todo{Kapitel 3.1.2: Code erläutern}{}
		\item Lineares Modell Theorie (Optionen?)
		\Todo{Kapitel 3.1.3: Lineares Modell übernehmen}{}
		\Todo{Kapitel 3.1.3: verwendete Umsetzung und Formeln erklären}{}
		\Todo{Kapitel 3.1.3: Grafiken ergänzen}{}
		\item Lineares Modell Implementierung --> selfmade
		\Todo{Kapitel 3.1.4: Implementierung beenden}{}
		\Todo{Kapitel 3.1.4: Code erläutern}{}
	\end{itemize}	
	\item Erklärung im speziellen Counterfactual
	\Todo{Kapitel 3.2: Counterfactual anlegen}{}
	\begin{itemize}
		\item Coutnerfactuals FAT-Forensics Theorie
		\Todo{Kapitel 3.2.1: Counterfactuals übernehmen}{}
		\Todo{Kapitel 3.2.1: Counterfactuals verwendete Formeln (+ brute force statt Loss function) erklären}{}
		\Todo{Kapitel 3.2.1: Grafiken ergänzen}{}
		\item Coutnerfactuals FAT-Forensics Implementierung --> Übernahme und Verwendung
		\Todo{Kapitel 3.2.2: Implementierung beenden}{}
		\Todo{Kapitel 3.2.2: Code erläutern}{}
		\item DiCE Theorie
		\Todo{Kapitel 3.2.3: DiCE übernehmen}{}
		\Todo{Kapitel 3.2.3: DiCE verwendete Formeln erklären}{}
		\item DiCE Implementierung --> Übernahme und Verwendung
		\Todo{Kapitel 3.2.4: Implementierung beenden}{}
		\Todo{Kapitel 3.2.4: Code erläutern}{}
	\end{itemize}
	\item Erklärung im speziellen Feature Contribution
	\Todo{Kapitel 3.3: Feature Contribution anlegen}{}
	\begin{itemize}
		\item DeepSHAP Theorie
		\Todo{Kapitel 3.3.1: Shap, 'DeepLIFT übernehmen}{}
		\Todo{Kapitel 3.3.1: Shap, DeepLIFT, DeepSHAP Formeln erklären}{}
		\Todo{Kapitel 3.3.1: Grafiken ergänzen}{}
		\item DeepSHAP Implementierung --> Übernahme und Verwendung
		\Todo{Kapitel 3.3.2: Implementierung beenden}{}
		\Todo{Kapitel 3.3.2: Code erläutern}{}
		\item LRP Theorie
		\Todo{Kapitel 3.3.3: LRP übernehmen (Proposal \& Proseminar)}{}
		\Todo{Kapitel 3.3.3: LRP Formeln erklären}{}
		\Todo{Kapitel 3.3.3: Grafiken ergänzen}{}
		\item LRP Implementierung --> selfmade
		\Todo{Kapitel 3.3.4: Implementierung beenden}{}
		\Todo{Kapitel 3.3.4: Code erläutern}{}
	\end{itemize}
	\item Metadaten
	\Todo{Kapitel 3.4: Metadaten ergänzen}{}
	\item Webpage
	\Todo{Kapitel 4: Webpage Ein- und Überleitung, Begründung für Nutzung }{}
	\begin{itemize}
		\item Design Idee
		\Todo{Kapitel 4: Webpage entwerfen}{}
		\Todo{Kapitel 4: Webpage einbauen}{}
		\Todo{Kapitel 4: Webpage erläutern}{}
		\item Theorie, verwendete Grundlagen
		\Todo{Kapitel 4: verwendete Code Basis / Theorie erkläutern}{}
		\item Implementierung / Umsetzung
		\Todo{Kapitel 4: eigenen Code erklären}{}
	\end{itemize}
	\item Ergebnis
	\Todo{Kapitel 5: Ergebnis formulieren}{}
	\item Fazit
	\Todo{Kapitel 5: Fazit (und Ausblick ?)}{}
	
	
	\Todo{Allgemein: Referenzen sortieren, aufhübschen und einbinden}{}
	\item Abbildungen
	\item Algorithmen
	\item Code
	\item Literatur
\end{itemize}

